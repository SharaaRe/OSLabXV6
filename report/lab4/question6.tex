هنگام کامپایل لینوکس، یکی از مهم‌ترین تنظیمات به تنظیم 
\lr{SMP} 
مربوط می‌شود. در صورتی که سیستم عامل قرار است فقط با یک هسته کار کند، می‌توانیم این تنظیم را غیرفعال کنیم. چرا که بسیاری از مشکلات قفل‌گذاری در سیستم‌های چندهسته‌ای رخ می‌دهند. با غیرفعال کردن این تنظیم می‌توان از اضافه شدن کدهای غیرلازم به تصویر لینوکس جلوگیری کرد. برای مثال در سیستم‌های تک‌هسته‌ای دیگر نیازی به قفل چرخشی وجود ندارد. پس می‌توان از سربار ناشی از آن خلاص شد.

در سیستم عامل لینوکس با ریزدانه 
\LTRfootnote{Fine-grained} 
کردن قفل‌ها مقایس پذیری تامین شده است. ریزدانه کردن به این صورت است که برای مثال یک لیست پیوندی را در نظر بگیرید. اگر برای کل این لیست از یک قفل استفاده کنیم، پردازنده‌های زیادی منتظر خواهند بود و سربار پروزرسانی این لیست نیز زیاد می‌شود. اما ممکن است پردازنده‌ی دارای قفل فقط یک یا چند گره از این لیست را په روز کند. پس می‌توان با ریز دانه کردن قفل، برای هر گره از یک قفل استفاده کرد. این کار اجازه فعالیت هم‌زمان به پردازنده‌هایی که با گره یکسانی کار ندارند، می‌دهد و از سربار په روز رسانی لیست کم می‌کند. 

اما ریزدانه کردن قفل همیشه به نفع نیست. اگر رقابت زیادی برای گرفتن قفل وجود نداشته باشد، ریزدانه کردن سربار جدیدی به سیستم اضافه می‌کند. می‌توان با استفاده از تنظیمات لینوکس هنگام کامپایل به نتیجه‌ی مطلوبی دست یافت.