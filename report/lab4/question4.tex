در سطح سخت افزار دو نوع از راه‌حل‌های موجود برای انسجام حافظه‌ی نهان 
\LTRfootnote{Cache coherence}، 
\lr{Directory Protocol} 
ها و 
\lr{Snoopy Protocol} 
ها هستند. تفاوت راه‌حل‌های سخت‌افزاری در مقایسه با راه‌حل‌های نرم‌افزاری در زمان رسیدگی به ناسازگاری حافظه‌ی نهان 
\LTRfootnote{Cache inconsistency} 
است. در راه‌حل‌های سطح سخت‌افزار بعد از اتفاق یک ناسازگاری میان حافظه‌ی نهان هسته‌های مختلف، وضعیت داده‌ی حافظه‌های نهان بررسی می‌شود که این خود تاثیر مثبتی در استفاده بهینه و با سریار پایین از حافظه‌ی نهان دارد.

در 
\lr{Directory Protocol} 
ها برای نگه‌داری وضعیت حافظه‌ی نهان هسته‌ها از یک کنترلر مرکزی درون کنترلر حافظه‌ی اصلی استفاده می‌شود. هرگاه یکی از هسته‌ها  (از طریق کنترلر حافظه‌ی نهان خود) متقاضی دسترسی به داده‌ای در حافظه‌ی اصلی یا در حافظه‌ی نهان دیگر هسته‌ها باشد، این درخواست را به کنترلر مرکزی ارسال می‌کند. بدین ترتیب مدیریت داده‌های مشترک توسط کنترلر مرکزی انجام می‌شود که از بروز ناسازگاری در آن‌ها جلوگیری می‌شود. همچنین همه‌ی هسته‌ها موظف هستند وضعیت به روز را از کنترلر مرکزی دریافت کنند.

در 
\lr{Snoopy Protocol} 
ها مسولیت انسجام حافظه‌ی نهان میان کنترلر حافظه‌ی نهان هسته‌ها توزیع می‌شود. هر کنترلر که متوجه تغییری در داده‌ی مشترک بین حافظه‌ی نهان خود و دیگر حافظه‌ها شد، از طریق ارسال پیامی به دیگر کنترلر‌ها، این تغییر را اطلاع رسانی می‌کند. برای ارسال این پیام دو رویکرد کلی وجود دارد. 
\lr{Write-invalidate} و \lr{Write-update}. 
در رویکرد اول، یکی از کنترلرها با ارسال پیام نامعتبر سازی داده‌ی مشترک، داده را به صورت انحصاری به اختیار می‌گیرد و می‌تواند آن را به روز کند و حافظه‌های دیگر نیز می‌توانند با ارسال درخواست، داده‌ی به روز را دریافت کنند. از این رو در هر لحظه فقط یک نویسنده می‌توانیم داشته باشیم. در رویکرد دوم تغغیر در داده‌ی مشترک به صورت پیام به روز رسانی به دیگر حافظه‌ها ارسال می‌شود. از این رو می‌توانیم چندین نویسنده در هر لحظه داشته باشیم.