زمان‌بندهای بی‌درنگ لینوکس از سیاست‌های
\lr{SCHED\_FIFO}،
\lr{SCHED\_RR}
و
\lr{SCHED\_DEADLINE}
استفاده می‌کنند. این سیاست‌ها مستقل از
\lr{CFS}
مدیریت می‌شوند. تسک‌های موجود در صف‌های این دو همواره با اولویت بالاتری نسبت به تسک‌های صف CFS
(\lr{SCHED\_NORMAL})
اجرا می‌شوند. یعنی تا وقتی‌که پردازه‌ی قابل اجرا در صفوف بی‌درنگ موجود باشد، پردازه‌های صف عادی اجرا نمی‌شوند. هنگامی‌که یک پردازه‌ی بی‌درنگ شروع به اجرا می‌شود، تا زمانی ادامه می‌دهد که بلوکه شود یا به صورت داوطلبانه پردازنده را تحویل دهد. در این صفوف اولویت بندی به صورت
\lr{static}
انجام می‌شود و همواره پردازنده‌ی پراولویت‌تر می‌تواند پردازنده را از پردازه‌ی کم‌اولویت‌تر قبضه کند و پردازه‌ی کم‌اولویت‌تر هیچ‌گاه نمی‌تواند پردازه‌ی پراولویت‌تر را قبضه کند.