توازن بار نباید بر اساس تعداد ریسه‌ها انجام گیرد چون در این صورت اولویت بندی بی‌اساس خواهد بود و ممکن است ریسه‌های کم اولویت و پراهمیت به مقدار یکسان توان پردازشی دریافت کنند. اما اولویت‌بندی تنها بر اساس وزن ریسه‌ها نیز کافی نیست، چون در این شرایط ممکن است هسته‌ای که مشغول به ریسه‌های پراهمیت است، در زمان های بی‌کاری پردازه‌های خود، پردازه‌های سایر هسته‌ها را بدزدد. یک الگوریتم توازن بار مناسب، تا حد امکان از این اتفاق پرهیز می‌کند، چون دزدیدن ریسه ها در تناقض با هدف تخصیص صف‌های جدا برای هسته است. فی الواقع ریسه‌های پراولویت‌تر لزوما به توان پردازشی بیشتری نیاز ندارند.

با توجه به توضیحات فوق، عملیات توازن بار که در الگوریتم زمان‌بندی کاملا عادلانه استفاده می‌شود، از کمیتی به نام بار
\lr{load}
استفاده می‌کند. این کمیت هم با وزن ریسه‌ی متناظر و هم با پردازش مورد نیاز آن رابطه‌ی مستقیم دارد. هسته‌های
\lr{core}
پردازنده بنا به منابعی که با هم به اشتراک گذاشته‌اند، تشکیل یک ساختار سلسله مراتبی
\lr{hierarchical structure}
می‌دهند که در پایین این ساختار، خود این هسته‌ها قرار دارند. هسته‌هایی که منابع مشترک دارند، لایه‌های بالاتر این سلسله مراتب را تشکیل می‌دهند. به هر یک از این لایه‌ها، یک حوزه‌ی زمان‌بندی
\lr{scheduling domain}
گفته می‌شود. در زمان توزیع بار در هر حوزه‌ی زمان‌بندی، یک هسته انتخاب شده و این هسته بار تمامی گروه‌های زمان‌بندی (هسته‌های موجود در لایه‌ی پایین‌تر ساختار) موجود در این حوزه را حساب می‌کند. سپس مشغول‌ترین گروه را بر اساس وزنی که حمل می‌کند انتخاب می‌کند. اگر بار این گروه از گروه محلی این هسته (هسته‌ی توزیع کننده ی بار) بیشتر بود، با دزدیدن پردازه‌ها بار توزیع می‌شود در غیر این صورت بار در سطح فعلی، منصفانه در نظر گرفته می‌شود و توزیع بار در این مرحله موقتا خاتمه می‌یابد، تا همین عملیات در توزیع بار بعدی تکرار گردد.