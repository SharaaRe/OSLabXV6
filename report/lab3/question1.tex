تابع
\lr{scheduler()}
هنگام اجرا کردن یک پردازه‌ی جدید، ضمن تغییر وضعیت پردازه‌ی مذکور، متن موجود روی پردازنده و متن پردازه را تعویض می کند. یعنی متن مربوط به اجرای تابع
\lr{scheduler()}
به جای محتوای اصلی پردازه در متغیر
\lr{p->context}
ذخیره می گردد. در ادامه پردازه اجرا می‌شود تا به هر طریق دستور
\lr{sched()}
اجرا گردد. در طی این اجرا نهایتا با دستور
\begin{lstlisting}[language=c++]
swtch(&p->context , mycpu()->scheduler);
\end{lstlisting}
تعویض متن انجام گرفته و محتوای مربوط به زمان‌بند در اختیار پردازنده قرار میگیرد تا ادامه ی کار آن انجام گیرد.