\begin{itemize}
	\item [8]
	\lr{dwrf} 
	قالب دودویی برای اشکال‌زدایی است. هنگام کامپایل با 
	\lr{-g} 
	سیمبول‌های مورد نیاز برای کمک به اشکال‌زدایی و فهم کد باینری تولید می‌شوند.
	این دستور فایل‌های مربوط به اشکال‌زدایی را درصورت وجود قبل از نمایش دیکامپرس کرده و نمایش می‌دهند.
	\lr{Info} 
	اطلاعات مربوط به فایل 
	\lr{.debug\_info} 
	را نمایش می‌دهد.
	\lr{Decodedline} 
	محتوای ترجمه شده‌ی فایل 
	\lr{.debug\_line} 
	را نمایش می‌دهد.
	
	\item [9]
	برای اجرای کد c نیاز به محیط نرم‌افزاری است که ازجایی که هنگام بوت کردن این محیط از طرف سیستم‌عامل محیا نیست باید از جای دیگری به صورت حداقلی برای ارتباط کد c و محیط سخت‌افزاری وجود داشته باشد. به همین دلیل یک کد اسمبلی برای تهیه‌‌ی این محیط هنگام بوت کردن باید وحود داشته باشد که 
	\lr{bootasm.S} 
	این کد اسمبلی است.
	\item [11]
	تا قبل از ارایه‌ی میکروپروسسور ۸۰۲۸۶ که مود پروتکتد را معرفی کرد تنها مود موجود برای پردازنده‌های 
	\lr{x86} 
	مود حقیقی بوده و برای سازگاری با ورژن‌های قدیمی‌تر تمام پردازنده‌های 
	\lr{x86} 
	در مود حقیقی شروع می‌شوند. درواقع دز هنگام بوت با شناسایی محیط پردازنده می‌تواند خود را به مودهای دیگر ببرد.
	در مود حقیقی هر برنامه‌ای می‌تواند به همه‌ی حافظه دسترسی داشته باشد و به همین دلیل می‌توانند باعث اختلال درکار هم دیگر شوند و عملا مالتی‌تسکینگ امکان‌پذیر نبوده است که در مود پروتکتد این مشکل با ایجاد سگمنت برای برنامه‌ها حل شده است.
	
\end{itemize}