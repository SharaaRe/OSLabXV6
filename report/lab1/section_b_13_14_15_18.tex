\begin{itemize}
	\item [13]
	برنامه‌ی بوت‌لودر هسته‌ی 
	\lr{xv6} 
	را در آدرس فیزیکی 
	\lr{0x100000} 
	کپی می‌کند. دلیل اینکه هسته در آدرس 
	\lr{0x80100000} 
	کپی نمی‌شود، این است که ممکن است ماشین‌های کوچکی وجود داشته باشند که آدرس فیزیکی به این بزرگی نداشته باشند. همچنین هسته در آدرس 
	\lr{0x0} 
	هم کپی نمی‌شود. چون آدرس 
	\lr{0xa0000} 
	تا 
	\lr{0x100000} 
	متعلق به دسنگاه‌های 
	\lr{I/O} 
	است.
	\item [14]
	این کد در هسته‌ی سیستم عامل لینوکس از فایل 
	\lr{arch/x86/boot/header.S} 
	شروع می‌شود.
	\item [15]
	نگهداری آدرس مجازی در ثبات کنترلی 
	\lr{cr3} 
	منطقی نیست. چون سخت افزار صفحه‌بند هنوز جدول صفحه ندارد و از چگونگی نگاشت آدرس‌های مجازی به فیزیکی آگاه نیست.
	\item [18]
	چون نمی‌توان توصیفگر کد را برای هسته و کاربر به اشتراک گذاشت. چون 
	\lr{CPU} 
	از ایجاد وقفه از 
	\lr{CPL=0} 
	به 
	\lr{CPL=3} 
	جلوگیری می‌کند.
\end{itemize}