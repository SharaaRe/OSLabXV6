\begin{itemize}
	\item [2] 
	\textbf{گروه \lr{basic headers}}: 
	این گروه شامل فایل‌های سرایند اساسی است. مانند 
	\lr{\lstinline|types.h|} 
	که شامل تعاریف انواع داده‌ای مورد استفاده در کد برنامه است یا 
	\lr{\lstinline|param.h|} 
	که شامل پارامترهای اولیه (مانند انواع محدودیت‌ها: محدودیت تعداد پردازه‌ها، محدودیت تعداد CPUها و ...، اندازه‌ی بلوک‌های مختلف و ...) برای سیستم عامل است.
	
	\textbf{گروه \lr{entering xv6}}: 
	این گروه شامل فایل اصلی 
	\LTRfootnote{main} 
	برنامه و دو فایل اسمبلی دیگر برای شروع به کار سیستم عامل است. شروع کد سیستم عامل در فایل‌های این گروه قرار دارد.
	
	\textbf{گروه \lr{locks}}: 
	همان‌طور که از اسم این گروه معلوم است، پیاده سازی مربوط به قفل‌ها در این گروه قرار دارد.
	
	\textbf{گروه \lr{processes}}: 
	تمام پیاده سازی‌های مربوط به پردازه‌ها اعم از حافظه‌ی مجازی، جدول صفحه
	\LTRfootnote{page table}، 
	مدیریت حافظه‌ی فیزیکی، جابجا شدن بین پردازه‌ها و ... در این گروه قرار دارند.
	
	\textbf{گروه \lr{system calls}}: 
	در این گروه توابعی برای راحتی کار با فراخوانی‌های سیستمی نوشته شده است. همچنین رسیدگی به وقفه‌ها نیز در این گروه انجام می‌شود.
	
	\textbf{گروه \lr{file system}}: 
	همان‌طور که می‌دانیم، 
	\lr{xv6} 
	در معماری خود رابطی بسیار کاربردی به اسم فایل دارد که سطح انتزاع 
	\LTRfootnote{Abstraction level}
	خوبی برای عملیات 
	\lr{I/O} 
	فراهم می‌کند. پیاده سازی‌های مربوط به این رابط در این گروه قرار دارند.
	
	\textbf{گروه \lr{pipes}}: 
	پیاده سازی ویژگی پایپ (روشی برای ورودی و خروجی)
	
	\textbf{گروه \lr{string operations}}: 
	در این گروه توابعی مفید برای کار با رشته‌ها تعریف شده‌اند.
	
	\textbf{گروه \lr{low-level hardware}}: 
	پیاده سازی‌های سطح پایین برای دسترسی به سخت افزار در این گروه قرار دارند. (ارتباط با سریال، پردازش‌های چند پردازه‌ای و ...)
	
	\textbf{گروه \lr{user-level}}: 
	برنامه‌های کاربردی سطح کاربر در این گروه قرار دارند. (مانند برنامه پوسته 
	\LTRfootnote{Shell})
	
	\textbf{گروه \lr{bootloader}}: 
	برنامه‌ی 
	\lr{boot loader} 
	برای شروع به کار اولیه (هسته‌ی) سیستم عامل (کپی کردن سیستم عامل در حافظه)
	
	\textbf{گروه \lr{link}}: 
	اسکریپتی برای لینکر هسته
	
	\textbf{در سیستم‌عامل لینوکس} 
	فایل‌های هسته‌ی سیستم عامل در پوشه‌ی 
	\lr{kernel}، 
	فایل‌های سرایند در پوشه‌ی 
	\lr{include} 
	و فایل‌سیستم در پوشه‌ی 
	\lr{fs} 
	قرار دارند.
	\item [3]
	خروجی دستور 
	\lr{\lstinline|make -n|}: 
	\begin{latin}
		\lstinputlisting[language=bash,caption=make -n,label=maken]{maken.l}
	\end{latin}
	با توجه به 
	\ref{maken} 
	بعد اجرای خط ۱ فایل‌های باینری ساخته می‌شوند و با اجرای خط ۲ این فایل‌ها به لینک شده و فایل اصلی هسته را می‌سازند.
	\item [6]
	با توجه به سه آخر 
	\ref{maken}، 
	ابتدا ۱۰ هزار بلوک اول دیسک 
	\lr{xv6} 
	را صفر می‌کند. سپس در سکتور اول فایل بوت (فایل 
	\lr{bootblock}) 
	را کپی کرده و با اجرای خط آخر، هسته را در سکتور دوم کپی می‌کند.
\end{itemize}