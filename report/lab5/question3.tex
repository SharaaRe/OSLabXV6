ساختار یک PTE از بخش‌های مختلفی تشکیل شده شده که بخش اول آن مربوط به شماره‌ی frame می‌باشد. (تعداد بیت‌های تخصیص داده شده برای این بخش به تعداد‌ رده‌های صفحه‌بندی وابسته است.) غیر از این بخش چندین بخش به عنوان flag نیز نگه‌داری می‌شود. این flag ها را در ادامه توضیح می‌دهیم:

\begin{description}
	\item[\lr{Present/Absent bit}] 
	این بیت نشانده‌ی حاضر بودن/نبودن صفحه‌ی مورد نظر است. در صورت حاضر نبودن 
	\lr{Page Fault} 
	رخ می‌دهد. بدین معنی که صفحه‌ی مورد نظر در حافظه وجود ندارد.
	\item[\lr{Protection bit}] 
	این بیت مجوزهای صفحه‌ی مورد نظر را تعیین می‌کند. 
	\lr{(read write execute)}
	\item[Referenced bit] 
	این بیت نشان‌دهنده‌ی این است که آیا این صفحه در سیکل قبلی استفاده شده یا خیر.
	\item[Caching enabled/disabled] 
	در صورت غیرفعال بودن، داده‌ی مورد نیاز همیشه از حافظه‌ی اصلی خوانده می‌شود. در مواقعی نیاز است که امکان دارد مقدار کش شده قدیمی شود و نیاز به دریافت مقدار جدید از حافظه باشد.
	\item[Modified bit] 
	وقتی مجوز نوشتن در صفحه لازم است، این بیت فعال می‌شود تا تغییرات ثبت شوند.
\end{description}