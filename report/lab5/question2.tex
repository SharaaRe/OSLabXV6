وقتی از صفحه‌بندی چندرده‌ای (سلسله مراتبی) استفاده می‌کنیم، دیگر نیاز به ساختن و ذخیره‌ی جدول صفحه‌ی مربوط به قسمت‌هایی از حافظه که توسط پردازه استفاده نمی‌شوند، نیست. همچنین می‌توان قسمت‌هایی از جدول صفحه را که در حال حاضر مورد نیاز پردازه نیست، می‌توان swap out کرد و در دسیک نگه داشت. از این رو مقدار حافظه‌ی مصرفی در صفحه‌بندی سلسله مراتبی کاهش می‌یابد.
این سناریو را در نظر بگیرید که جدول صفحه دو رده داشته باشد. 
\begin{latin}
	\lstinputlisting{page_table_ex.txt}
\end{latin}

در این مثال چهار آدرس داریم که به ۲ ختم می‌شود. (یعنی ۶/۲، ۷/۲، ۸/۲ و ۹/۲) پس برای هر مدخل در جدول داخلی چهار آدرس موجود است و در کل ۱۶ آدرس. اما به جای ذخیره کردن ۱۶ مدخل، فقط ۸ مدخل ذخیره شده است.