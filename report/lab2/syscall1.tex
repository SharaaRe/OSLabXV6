برای اضافه کردن یک فراخوانی سیستمی به 
\lr{xv6}
 تعدادی از فایل‌ها باید تغییر کنند. از جمله 
\lr{user.h}
که تعریف سیستم‌کال باید به آن اضافه شود. 
\lr{usys.S}
که ماکرو موجود در آن باید سیستم‌کال جدید را هم اجرا کند. 
\lr{defs.h}
 که تعاریف همه‌ی توابع و دیگر متغیرهای لازم را دارد. 
\lr{sysproc.c}
که تابع پیاده سازی شده در 
\lr{proc.c}
 را فراخوانی می‌کند و در صورت لزوم آرگومان‌ها را از روی پشته می‌خواند. در این فراخوانی باید آرگومان از طریق رجیستر فرستاده شود که در ابتدا مقدار رجیستر 
\lr{ebx}
 بر روی یک متغیر نگه‌داری می‌شود و آرگومان بر روی آن نوشته می‌شود. بعد از خوانده شدن در تابع فراخوانی سیستمی و استفاده و بازگشت مقدار قبلی نگه‌داری شده در متغیر دوباره بر روی رجیستر 
 \lr{ebx}
  نوشته می‌شود.