برای پیاده‌سازی سیستم کال خواسته شده، باید ابتدا در ساختار داده‌ی
\lr{proc}
آرایه‌ای از
\lr{struct sysclog}
اضافه کنیم که هر عضو این آرایه، در حقیقت یک زوج مرتب
\lr{(system call number, return value)}
می‌باشد. پیاده‌سازی ساختار داده‌ی
\lr{sysclog}
و تغییرات اعمال شده در فایل
\lr{proc.h}
انجام شده است. در فایل
\lr{syscall.c}
بعد از فراخوانی هر سیستم‌کال، شماره‌ی آن‌ را از
\lr{trap}
مربوطه و خروجی سیستم‌کال را در آرایه‌ی ایجاد شده ذخیره می‌کنیم. سیستم‌کال
\lr{print\_syscalls}
با پیمایش روی
\lr{ptable}
زوج‌های
\lr{sysclog}
را به‌ازای هر پردازه‌ی فعال چاپ می‌کند. تنها نکته‌ی باقی‌مانده، سیستم‌کال‌های مربوط به پردازه‌های
\lr{terminate}
شده است. برای نگه‌داری رکورد سیستم‌کال‌های این پردازه‌ها، آرایه‌ای مشابه
\lr{ptable}
به‌نام
\lr{pptable}
ساخته شد و اطلاعات پردازه‌های
\lr{exit} یا \lr{kill}
شده در این جدول اضافه گشت. فراخوانی
\lr{print\_syscalls}
اطلاعات هر دو جدول را چاپ می‌کند.